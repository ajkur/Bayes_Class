\documentclass[11pt]{article}
\usepackage{amsmath,amsfonts,amsthm,amssymb}
\usepackage{times}
\usepackage[pdftex]{graphicx}
\usepackage[pdftex,
        colorlinks=true,
        urlcolor=linkblue,     % \href{...}{...} external (URL)
        citecolor=linkred,     % citation number colors
        linkcolor=linknavy,    % \ref{...} and \pageref{...}
        pdftitle={Flame Spread},
        pdfauthor={Andrew Kurzawski},
        pdfsubject={Flame Spread},
        pdfkeywords={UT},
        pdfproducer={pdflatex},
        pagebackref,
        pdfpagemode=UseNone,
        bookmarksopen=true,
        plainpages=false]{hyperref}
\usepackage{pdfsync}
\usepackage{color}
\usepackage{titling}
\usepackage[nottoc,notlof,notlot]{tocbibind} % Put the bibliography and index in the ToC
\usepackage{listings}

\definecolor{linknavy}{rgb}{0,0,0.50196}
\definecolor{linkred}{rgb}{1,0,0}
\definecolor{linkblue}{rgb}{0,0,1}

\setlength{\droptitle}{-4em}     % Eliminate the default vertical space on the title page.
\addtolength{\droptitle}{5cm}   % Only a guess. Use this for adjustment of the title placement.

\setlength{\textwidth}{6.5in}
\setlength{\textheight}{9.0in}
\setlength{\topmargin}{0.in}
\setlength{\headheight}{0.in}
\setlength{\headsep}{0.in}
\setlength{\parindent}{0.25in}
\setlength{\oddsidemargin}{0.0in}
\setlength{\evensidemargin}{0.0in}

% Python Setup
\lstset{
  language=Python,                % choose the language of the code
  basicstyle=\footnotesize,       % the size of the fonts that are used for the code
  numbers=left,                   % where to put the line-numbers
  numberstyle=\footnotesize,      % the size of the fonts that are used for the line-numbers
  stepnumber=1,                   % the step between two line-numbers. If it is 1 each line will be numbered
  numbersep=5pt,                  % how far the line-numbers are from the code
  backgroundcolor=\color{white},  % choose the background color. You must add \usepackage{color}
  showspaces=false,               % show spaces adding particular underscores
  showstringspaces=false,         % underline spaces within strings
  showtabs=false,                 % show tabs within strings adding particular underscores
  frame=single,           % adds a frame around the code
  tabsize=2,          % sets default tabsize to 2 spaces
  captionpos=b,           % sets the caption-position to bottom
  breaklines=true,        % sets automatic line breaking
  breakatwhitespace=false,    % sets if automatic breaks should only happen at whitespace
  escapeinside={\%*}{*)}          % if you want to add a comment within your code
}

% Uncomment lines and make changes to have a header
\newcommand{\code}[1]{ % change to [2]
  % \hrulefill
  % \subsection*{#1}
  \lstinputlisting{#1} % change to {#2}
  \vspace{2em}
}

\title{Quantifying Flame Spread Uncertainty Using Bayesian Methods}
\author{
        \\
        Andrew Kurzawski \\
        Bayesian Statistical Methods \\ 
        Semester Project \\
}
\date{May 3, 2013}

\begin{document}

\maketitle

\clearpage

\pagestyle{plain}

% Project Proposal: Model Calibration and Uncertainty Quantification Using Bayesian Methods

\section{Introduction and Background}

Modeling physical systems requires that we prescribe values for the model parameters based on previous knowledge, correlations, engineering estimations, etc. If we possess expermental data for the system of interest, we can numerically invert for the unknown model parameters. In higher order systems, the solution space may be complicated, and inversion methods can easily get stuck in local minima and maxima. Bayesian inference offers a way to calibrate models of physical systems and calculate the uncertainty in model parameters by using an MCMC sampler.

Talk about grass fire: prediction problem, resource allocation problem, make analogy to huricane forecasting bands with 95 percent confidence interval.

Describe experiment, field size, data loggers, measured quantities.

Describe modeling possibilities and end with decision to use rothermal (just expand upon the email for most of this.) Like a bayesian regression where regression parameters have physical meaning, can be easily checked for non-physical predictions.


\section{Flame Spread Model Overview}

Lay out the equations as succinctly as possible. Cite Rothermal. 11 parameters, system of algebriac equations.


\section{Bayesian Model}

Describe data, spread, moisture etc.

Describe unknown parameters. describe priors (uniform for first cut, if time allows consider estimate betas or normals from literature(expert opinions))

Paste in some code maybe.


\section{Results and Analysis}

Describe tuning process. Aim for good mixing and low autocorrelation.

Posterior output. Did it fit? Are the values physically significant? What is the uncertainty? What is the physical interpretation of this uncertainty (ie do a simple calculation to see the time frame in which fire could spread over a whole field to give a sense of scale)? What about outliers (check estimated mc data points)? Conclude with flame spread versus moisture content to show uncertainty and 95 percent probability interval. Explain use for extrapolating to other moisture contents, more data will decrease uncertainty.

Conclude with possibility of extension to more complicated forecasting models for resource allocation.

\clearpage
\appendix
\section{Appendix: PyMC Model Code}

% \bibliographystyle{unsrt}
% \bibliography{./flameBayesBiblio}

\end{document}

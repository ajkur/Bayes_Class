\documentclass[11pt]{article}
\usepackage{amsmath,amsfonts,amsthm,amssymb}
\usepackage{times}
\usepackage[pdftex]{graphicx}
\usepackage[pdftex,
        colorlinks=true,
        urlcolor=linkblue,     % \href{...}{...} external (URL)
        citecolor=linkred,     % citation number colors
        linkcolor=linknavy,    % \ref{...} and \pageref{...}
        pdftitle={Flame Spread},
        pdfauthor={Andrew Kurzawski},
        pdfsubject={Flame Spread},
        pdfkeywords={UT},
        pdfproducer={pdflatex},
        pagebackref,
        pdfpagemode=UseNone,
        bookmarksopen=true,
        plainpages=false]{hyperref}
\usepackage{pdfsync}
\usepackage{color}
\usepackage{titling}
\usepackage[nottoc,notlof,notlot]{tocbibind} % Put the bibliography and index in the ToC
\usepackage{listings}

\definecolor{linknavy}{rgb}{0,0,0.50196}
\definecolor{linkred}{rgb}{1,0,0}
\definecolor{linkblue}{rgb}{0,0,1}

\setlength{\droptitle}{-4em}     % Eliminate the default vertical space on the title page.
\addtolength{\droptitle}{5cm}   % Only a guess. Use this for adjustment of the title placement.

\setlength{\textwidth}{6.5in}
\setlength{\textheight}{9.0in}
\setlength{\topmargin}{0.in}
\setlength{\headheight}{0.in}
\setlength{\headsep}{0.in}
\setlength{\parindent}{0.25in}
\setlength{\oddsidemargin}{0.0in}
\setlength{\evensidemargin}{0.0in}

% Python Setup
\lstset{
  language=Python,                % choose the language of the code
  basicstyle=\footnotesize,       % the size of the fonts that are used for the code
  numbers=left,                   % where to put the line-numbers
  numberstyle=\footnotesize,      % the size of the fonts that are used for the line-numbers
  stepnumber=1,                   % the step between two line-numbers. If it is 1 each line will be numbered
  numbersep=5pt,                  % how far the line-numbers are from the code
  backgroundcolor=\color{white},  % choose the background color. You must add \usepackage{color}
  showspaces=false,               % show spaces adding particular underscores
  showstringspaces=false,         % underline spaces within strings
  showtabs=false,                 % show tabs within strings adding particular underscores
  frame=single,           % adds a frame around the code
  tabsize=2,          % sets default tabsize to 2 spaces
  captionpos=b,           % sets the caption-position to bottom
  breaklines=true,        % sets automatic line breaking
  breakatwhitespace=false,    % sets if automatic breaks should only happen at whitespace
  escapeinside={\%*}{*)}          % if you want to add a comment within your code
}

% Uncomment lines and make changes to have a header
\newcommand{\code}[1]{ % change to [2]
  % \hrulefill
  % \subsection*{#1}
  \lstinputlisting{#1} % change to {#2}
  \vspace{2em}
}

\title{Quantifying Flame Spread Rate Uncertainty Using Bayesian Methods}
\author{
        \\
        Andrew Kurzawski \\
        Bayesian Statistical Methods \\ 
        Semester Project \\
}
\date{May 3, 2013}

\begin{document}

\maketitle

\clearpage

\pagestyle{plain}


\section{Introduction and Background}

Modeling physical systems requires that we prescribe values for the model parameters based on previous knowledge, correlations, engineering estimations, etc. If we possess experimental data for the system of interest, we can numerically invert for the unknown model parameters. In higher order systems, the solution space may be complicated, and inversion methods can easily get stuck in local minima and maxima. The result of the inversion process is a set of point estimates for the model input parameters. However, it lacks the uncertainty information necessary for activities such as assessing the model's accuracy or conduction a risk assessment.

Bayesian inference offers a means to fit models of physical systems to observed data and calculate the uncertainty in the model parameters. This method assumes that we do not know the true distribution of the input parameters, but we can use observed data to reconstruct it. Given a scenario or collection of scenarios with observed data, Bayesian methods allow us to sample from the true distributions of the data and unobserved quantities of interest (QoI, plural QoIs).

Prediction of fire spread is a major concern to wildland fire fighters who use this information construct tactical plans for suppressing and controlling fires. This involves the allocation of resources (personal, command stations, airtankers, etc.) at the appropriate time and location in situations where a poor prediction could cause property damage and put fire fighters and residents lives at risk.

Research efforts in wildland fire modeling have produced models for fire behaviour that range in complexity from algebraic equations to full fluid dynamics and combustion models. This project will focus on the Rothermal flame spread model \cite{rothermel1972mathematical} that was derived from a combination of fire experiment correlations and physical principles. The result is a set of algebraic equations with eleven input parameters that can be computed quickly. The traditional approach to using the Rothermal model is calculating the flame spread rate using prescribed input parameters based on the fuel type, wind speed, and ground slope. Using this model with Bayesian methods is equivocal to conducting a Bayesian regression where the regression parameters each have a physical meaning, and can therefore be easily examined for non-physical predictions.

The scenario used for this study involves a large scale (~30 acres) field burn conducted at Camp Swift near Bastrop in the Spring of 2011 before Texas saw one of the worst fire seasons on record. The field was composed mostly of Little Bluestem grass with pockets of other vegetation interspersed throughout. We instrumented the field with twenty data loggers to track flame spread and temperature. Flame spread rate in the downwind direction was calculated by the time it took the flame to travel from one data logger to the next divided by the distance. Additionally, data was collected on the moisture content of the grass and weather conditions such as ambient temperature, wind speed and direction, and relative humidity. Some of these values correspond to Rothermal model inputs and will be used as scenario parameters when estimating the likelihood of the observed flame spread rates.


\section{Flame Spread Model Overview}

The Rothermal Model is a system of equations with eleven input parameters that can be used to calculate flame spread rate. Some of these parameters can be measured in the field and others are often prescribed fixed values based on the fuel type (i.e. trees, brush, grass). See Appendix A for descriptions of the input parameters. The full equation for rate of spread, $R$, is as follows:

\begin{equation}
R = \frac{I_R\xi(1+\phi_W+\phi_S)}{\rho_b\epsilon Q_{ig}}
\label{eq:roth_main}
\end{equation}

Each variable in Equation~\ref{eq:roth_main} is a function of one or more other variables that may or may not be input parameters. To make this model more usable, we will simplify the equation in terms of the input parameters. The variables $\xi$, $\rho_b$, and $\epsilon$ are all constructs with limited physical meaning that are each a function of the fuel load ($w_O$), fuel depth ($\delta$), fuel density ($\rho_p$), and surface area to volume ratio ($\sigma$). 

\begin{equation}
\frac{\xi}{\rho_b\epsilon} = \hat{F}(w_O, \delta, \rho_p, \sigma)
\label{eq:roth_fun}
\end{equation}

\noindent where $\hat{F}$ is a combination of polynomial functions of the input parameters which reduces Equation~\ref{eq:roth_main} down to

\begin{equation}
R = \hat{F}(w_O, \delta, \rho_p, \sigma) I_R\frac{1+\phi_W+\phi_S}{Q_{ig}}
\label{eq:roth_fun2}
\end{equation}

$I_R$ is another non-physical construct that, in addition to being a function of $w_O$, $\delta$, $\rho_p$, and $\sigma$, is a function of five other physical parameters: the total and effective mineral contents ($S_T$ and $S_e$), the moisture content of the fuel ($M$) and the moisture content of extinction ($M_{ext}$), and the heat of combustion $h$. Of these, $S_T$ and $S_e$ will be held constant for this analysis.

Of the remaining parameters, the wind factor ($\phi_W$) is a function of wind speed ($U$), $w_O$, $\delta$, $\rho_p$, and $\sigma$. The slope factor ($\phi_S$) will be neglected for this analysis as the field in question was relatively flat. The final parameter ($Q_{ig}$) is only a function the moisture content. The result is a reduced Rothermal mode that depends on eight input parameters.

\begin{equation}
R = F(w_O, \delta, \rho_p, \sigma, M, M_{ext}, h, U)
\label{eq:roth_fun3}
\end{equation}

The following Bayesian modeling and analysis will seek to find the true distributions of as many of these parameters as possible using scenario data from the large scale field burn.

\section{Bayesian Model Formulation}

\begin{figure}[h]
\begin{center}
\includegraphics[width=3.0in]{./Figures/field}
\end{center}
\caption{Schematic of experimental field. The green dots are data loggers, and the red X represents a broken device.}
\label{fig:field} 
\end{figure}

Several models will be constructed to estimate the flame spread rate, but first we take an inventory of the available data. From each of the data loggers we have the time at which the flame reached the device. If we consider flame spread in the downwind direction and take only point-to-point observations of the travel time from one device to the next, then the twenty data loggers embedded in a grid on the grass field produce sixteen unique spread rates. Four measurements came from a boundary that could have been affected by nearby trees and will be discarded to focus on grass flame spread. Of the remaining data loggers, one malfunctioned and will also be discarded (Figure~\ref{fig:field}. The result is eleven flame spread rate observations that will be treated as independent assuming that the flame spread rate has reached a quasi-steady state by the time it arrives at the first row of data loggers.

Accompanying each flame spread observation is a pre-burn moisture content ($M$) measurement of the grass. These scenario values are calculated as the average moisture content between data logger $i$ and $i+1$. These will remain as point estimates for the purpose of this exercise, but it should be noted that a full model would include uncertainty information for variation in $M$ between the two points. There is the possibility that some of these estimations could be outliers, and the effect of discarding these points will be examined briefly using the best fitting model.

The parameters $\sigma$, $M_{ext}$, and $h$ do not have any scenario data and while recommended values are provided by the U.S. Forest Service \cite{scott2005standard} there is no uncertainty information. Prior probability density functions will be chosen for each parameter based on the recommended values and compared to diffuse priors. The remaining parameters in Equation~\ref{eq:roth_fun3} ($w_O$, $\delta$, $\rho_p$ and $U$) are irreducible and scenario specific. Unfortunately, these are only available on a field-wide level with limited uncertainty information. Similar to the missing moisture content scenario parameter, a full analysis would include uncertainty models for each of these parameters from field data however they will remain fixed in the following models (see Appendix~\ref{ap:table}).

Previous research was conducted with a higher order physics-based model focused on testing the sensitivity of flame spread rate to multiple input parameters \cite{overholtcharacterization}. The results of this study suggest that flame spread rate is most sensitive to $\sigma$ and $M$. Therefore, these parameters are the QoIs whose effect on the uncertainty of the flame spread rate is a desired output of the following analysis.

We may now begin constructing the statistical model for this problem. The likelihood function for each of the models is assumed to be normally distributed with the form

\begin{equation}
Y_i \sim \mathcal{N}(R_i,\tau^2),\quad i = 1,...,11
\label{eq:like}
\end{equation}

\noindent where $Y_i$ is an observation of the spread rate, $R_i$ is the spread rate calculated from Equation~\ref{eq:roth_fun3}, and the prior distribution for $\tau$ is

\begin{equation}
\tau \sim \mathcal{U}(0,100)
\label{eq:tau_prior}
\end{equation}

We will define four models with different sets of priors on $\sigma$, $M_{ext}$, and $h$. The first model represents the ``uninformed'' case, however we can not use truly uninformed priors because extreme values result in non-physical rates of spread. Likewise, combinations of extreme non-physical inputs can also produce a physical rate of spread. We must then limit each parameter to a wide physically realistic space. The most ``uninformed'' way to accomplish this is to choose wide uniform priors which is common practice for Bayesian modeling of physical thermal and fluid systems \cite{miki2012bayesian,cheung2011bayesian}. The priors for Model 1 are chosen as follows:

\begin{align}
\begin{array}{ccc}
M_{ext} &\sim& \mathcal{U}(0,1) \\
\sigma &\sim& \mathcal{U}(500,50000) \\
h &\sim& \mathcal{U}(1,50000)
\end{array}
\label{eq:mod1_priors}
\end{align}

The motivation choosing the following models will be discussed in Section~\ref{sec:res}, and for now we will simply describe their development. For Model 2, we will elicit prior information on the mean $M_{ext}$ from Scott, et al \cite{scott2005standard} and high estimate from ``expert'' opinion. The mean of the $M_{ext}$ values from literature is calculated to be 0.268 and it is estimated that there is a 99\% probability that $M_{ext}$ is less than 0.6. We choose to use a Gamma prior as $M_{ext}$ is a positive value without a clear upper bound. Using this information with the equations for the mean and cumulative distribution function we have two equations and two unknowns and can numerically solve for the parameters $\alpha$ and $\beta$, leading us to the formulation of Model 2.

\begin{align}
\begin{array}{ccc}
M_{ext} &\sim& \Gamma(5.6,20.87) \\
\sigma &\sim& \mathcal{U}(500,50000) \\
h &\sim& \mathcal{U}(1,50000)
\end{array}
\label{eq:mod2_priors}
\end{align}

Following a similar methodology used for $M_{ext}$, we will construct a prior for $\sigma$ eliciting data from previous measurements that have not been published. The mean is estimated to be 2300 $ft^{-1}$ with a 99\% probability that $\sigma$ is less than 3700 $ft^{-1}$, and we will choose a Gamma prior. However, at some point when $\sigma$ is less than 100 the model becomes unstable producing non-physical spread rates. To circumvent this issue, we will shift the elicited values down by 100 to create a prior and shift the samples up when evaluating the model effectively sampling from a Gamma distribution beginning at 100. Solving for the parameters of the Gamma distribution in the same manner as described previously, $\alpha$ and $\beta$ are 17.4 and 0.0079 respectively. The priors for Model 3 are as follows. 

\begin{align}
\begin{array}{ccc}
M_{ext} &\sim& \Gamma(5.6,20.87) \\
\sigma &\sim& \Gamma(17.4,0.0079) \\
h &\sim& \mathcal{U}(1,50000)
\end{array}
\label{eq:mod3_priors}
\end{align}

The final model will be constructed by choosing the best fitting model from the previous three using the deviance information criterion (DIC) and removing the outlying estimated moisture content data points. Note that an informed prior was not considered for the heat of combustion ($h$) as only one value given for all models \cite{scott2005standard} and there are a variety measurement techniques that can give significantly different results for the same type of fuel. Without a consistent data source, we can not bring in prior knowledge to improve upon the prescribed uniform distribution. 

These models will be coded in Python using the PyMC module for Bayesian inference and Markov chain sampling methods. All models will use the standard Metropolis-Hastings algorithm to draw samples. Additionally, model code will be provided with the report as an electronic attachment.


\section{Results and Analysis}
\label{sec:res}

To tune the models, we run 10000 iterations with no burn in and thinning to examine convergence and autocorrelation. We choose a value for burn-in based on when the chain appears to have converged to the true posterior which is evident when there is good mixing and no drift in the mean value. We then thin the samples such that there is low autocorrelation. Finally, we select the number of iterations such that each model will produce the same number of posterior samples. The values for each model can be found in Table~\ref{tab:tune}.

\begin{table}[h]
\caption{Tuning parameters for Models 1-3.}
\begin{center}
  \begin{tabular}{cccc}
    \hline\noalign{\smallskip}
    Model & Burn in & Thin & Iterations \\
    \noalign{\smallskip}\hline\noalign{\smallskip}
    1 & 8000 & 75 & 83000 \\
    2 & 6000 & 50 & 56000 \\ 
    3 & 2000 & 25 & 27000 \\ 
    \noalign{\smallskip}\hline
  \end{tabular}
\end{center}
\label{tab:tune}
\end{table}
\newpage
The goodness of fit is determined by calculating a p-value with Freeman-Tukey statistic \cite{brooks2000bayesian} provided in PyMC where  

\begin{equation}
D(R|M_{ext},\sigma,h) = \Sigma(\sqrt{R_j} - \sqrt{e_j})^2
\label{eq:D_val}
\end{equation}

\noindent and

\begin{equation}
p = Pr[D(R_{sim}|M_{ext},\sigma,h) > D(R_{obs}|M_{ext},\sigma,h)]
\label{eq:p_val}
\end{equation}

These values are listed in Table~\ref{tab:dic} along with the DIC for each fitted model. The p-values indicate that there is no lack of fit for each of the proposed models and the DIC indicates that Model 2 is the best of the three. To understand why this might be the case we visually examine the flame spread rate versus moisture content for values of $M_{ext}$, $\sigma$, and $h$ drawn from their posterior distributions.

\begin{table}[h]
\caption{DIC and p-values for Models 1-3. consider including some confidince intervals}
\begin{center}
  \begin{tabular}{ccc}
    \hline\noalign{\smallskip}
    Model & p-value & DIC \\
    \noalign{\smallskip}\hline\noalign{\smallskip}
    1 & 0.603 & 15.69 \\
    2 & 0.599 & 13.86 \\ 
    3 & 0.597 & 23.33 \\ 
    \noalign{\smallskip}\hline
  \end{tabular}
\end{center}
\label{tab:dic}
\end{table}

Figures~\ref{fig:m1_rate}-\ref{fig:m3_rate} show the flame spread rate versus moisture content where the blue dots are observed data from experiments; the green line is the spread rate calculated with the posterior mean values of $M_{ext}$, $\sigma$, and $h$; and the gray lines are replications sampled from the posterior distributions of the input parameters. The replications can be thought of as analogous to the uncertainty bands placed on the prediction of hurricane or storm movements given the output of a weather model. to We can see that there is high uncertainty in the spread rate for moisture content values between 0.4 and 0.5 (Figure~\ref{fig:m1_rate}), this is equivalent to flame spread occurring in very damp grass which we know to be physically unlikely. Upon further examination of the posterior output, we see that the 95\% probability interval for $M_{ext}$ is [0.22, 0.98] which is very large and includes physically unlikely values. This suggest the need to elicit prior information on $M_{ext}$.

\begin{figure}[h]
\begin{center}
\includegraphics[width=3.0in]{./Figures/mod_one/spread_rate}
\end{center}
\caption{Flame spread rate vs. moisture content for Model 1.}
\label{fig:m1_rate} 
\end{figure}

For Model 2, we have placed an informative prior on $M_{ext}$ which results in a decrease in the probability that flame spread will occur at high moisture content and better capturing of the observed data compared to Model 1 as indicated by the gray replication bands (Figure~\ref{fig:m2_rate}). We sought to bring in more information for model 3 by eliciting prior information on $h$, however we can see that the model does not capture outlying spread rates (Figure~\ref{fig:m3_rate}). This trend is also evident in the values of the DIC (Table~\ref{tab:dic}), which supports our observations that Model 2 is indeed the best choice. 

\begin{figure}[h]
\begin{center}
\includegraphics[width=3.0in]{./Figures/mod_two/spread_rate}
\end{center}
\caption{Flame spread rate vs. moisture content for Model 2.}
\label{fig:m2_rate} 
\end{figure}

\begin{figure}[h]
\begin{center}
\includegraphics[width=3.0in]{./Figures/mod_three/spread_rate}
\end{center}
\caption{Flame spread rate vs. moisture content for Model 3.}
\label{fig:m3_rate} 
\end{figure}

Now we use Model 2 and remove the outlying moisture content scenario parameters. This model will be referred to as Model 4. A quick calculation of the absolute residuals reveals that there are two possible outliers. Fitting the model yields a p-value of 0.581 indicating that there is no overall lack of fit. Using the DIC to compare Model 2 and Model 4 is not appropriate as they have different data sets, so we will examine the flame spread rate versus moisture content plot (Figure~\ref{fig:m4_rate}). For Model 4, all data points are included within the range of replication curves indicating a good fit to the data, however there is large uncertainty as the moisture content approaches zero indicating completely dry grass. 

\begin{figure}[h]
\begin{center}
\includegraphics[width=3.0in]{./Figures/mod_four/spread_rate}
\end{center}
\caption{Flame spread rate vs. moisture content for Model 4 (data with outliers removed).}
\label{fig:m4_rate} 
\end{figure}

% Model 4
% Bayesian p-value: p=0.581
% DIC 7.62

% Compare 95 percent posterior intervals of parameters to range tested in the naive sensitivity approach from firetech paper. Compare spread rates (add points to previous figures and make new ones). do this some other time, maybe for a paper

Prediction of flame spread is inherently a difficult problem due to the physics involved and it is of great importance during fire danger weather (low moisture content, high wind speeds). The data used for this study was not collected during fire danger weather and the inclusion of spread measurements at a higher range of moisture contents could help to decrease the uncertainty of the model predictions. To accurately assess the true uncertainty of the model, future experiments could include scenario data for the parameters that were held constant in this exercise. The true uncertainty is likely greater than the estimates presented in this report, however it offers a framework for continuing Bayesian analysis on larger and more robust fire ground data sets. This framework could then be extended to more complicated flame spread models with better representation of the physics than the Rothermal model.


\clearpage
\appendix
\section{Appendix: Rothermal Input Parameters}
\label{ap:table}

\begin{table}[h]
\caption{Input parameters for basic Rothermal model with values included for fixed parameters.}
\begin{center}
  \begin{tabular}{cccc}
    \hline\noalign{\smallskip}
    Parameter & Symbol & Value & Units \\
    \noalign{\smallskip}\hline\noalign{\smallskip}
    Dry Fuel Load & $w_O$ & 1.217 & $kg/m^2$ \\
    Fuel Depth & $\delta$ & 5 & $ft$ \\ 
    Surface Area to Volume Ratio & $\sigma$ & - & $1/ft$ \\ 
    Heat of Combustion & $h$ & - & $Btu/lb$ \\ 
    Dry Particle Density & $\rho_p$ & 512 & $kg/m^3$\\ 
    Moisture Content & $M$ & - & mass moisture/mass dry fuel \\ 
    Total Mineral Content & $S_T$ & 0.0555 & mass minerals/mass dry fuel \\
    Effective Mineral Content & $S_e$ & 0.01 & mass silica-free minerals/mass dry fuel \\
    Wind Velocity & $U$ & 2 & $m/s$ \\
    Slope Angle & $\phi$ & 0 & $rad$ \\
    Moisture Content of Extinction & $M_{ext}$ & - & mass moisture/mass dry fuel \\
    \noalign{\smallskip}\hline
  \end{tabular}
\end{center}
\label{tab:para}
\end{table}

\section{Appendix: PyMC Model Code}

Electronic attachment.

\bibliographystyle{unsrt}
\bibliography{./flameBayesBiblio}

\end{document}
